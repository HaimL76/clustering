\documentclass[12pt]{article}
\makeatletter
\newcommand*{\rom}[1]{\expandafter\@slowromancap\romannumeral #1@}
\makeatother
\usepackage{amsfonts}
\usepackage{amsmath}
\usepackage{tikz-cd}
\usepackage{xcolor}


\begin{document}


\textbf{Exercise}
Let \( \{ E_{i,j} \}_{i<j} \) be the set of all elementary matrices, of this form. \newline
Prove that \( E_{i,j}^{-1}=(b_{l,k}) \) is \( E_{i,j}=(a_{l,k}) \), when we substitute \( a_{i,j}=1 \) with \( b_{i,j}=-1 \) \newline

\textbf{Proof}
We can see that directly from the fact that if we multiply \( E_{i,j}^{-1} \) by \( E_{i,j} \) from the left 
then \( E_{i,j} \) is operating on \( E_{i,j}^{-1} \) by adding row \( j \) to row \(i \) \newline
So, in the product matrix, \( (c_{l,k}) \), in order to have \( 1 \) \newline 
on the main diagonal, we need them to exist on the main diagonal
of \( E_{i,j}^{-1} \), to begin with. Now, in order to have \( c_{i,j}=0 \), we need to have the addition of \( j \) to \( i \)
giving \( c_{i,j}=a_{i,j}+b_{i,j}=0 \Rightarrow b_{i,j}=-a_{i,j}=-1 \) \newline

\textbf{Exercise} Prove that if \( (a_{ij})=E_{i,j}, i<j \) is an elementary matrix, \newline
then \( \forall m \in \mathbb(N), E_{i,j}^m \) is \(E_{i,j} \), but with \( a_{ij}=m \) \newline
$$
	E_{i,j} = \begin{pmatrix} 
	1 & 0 & \dots & 0 & 0 \\
	0 & 1 & \dots & 0 & 0 \\
	0 & 0 & \ddots & 1 & 0 \\
	0 & 0 & \dots & 1 & 0 \\
	0 & 0 & \dots & 0 & 1 \\
	\end{pmatrix}
	\quad
	$$

$$ E_{i,j}^2=E_{i,j} \cdot E_{i,j} $$

Since \( E_{i,j} \) is en elemntary matrix, then it operates on the right matrix
as an addition of row \( j \) to row \( i \) \newline

So, $$
	E_{i,j}^2 = \begin{pmatrix} 
	1 & 0 & \dots & 0 & 0 \\
	0 & 1 & \dots & 0 & 0 \\
	0 & 0 & \ddots & 2 & 0 \\
	0 & 0 & \dots & 1 & 0 \\
	0 & 0 & \dots & 0 & 1 \\
	\end{pmatrix}
	\quad
	$$

We assume this is true for all \( E_{i,j}^m \), now we prove for \( E_{i,j}^{m+1} \)
$$
	E_{i,j}^{m+1} = E_{i,j} \cdot E_{i,j}^{m}=\begin{pmatrix} 
	1 & 0 & \dots & 0 & 0 \\
	0 & 1 & \dots & 0 & 0 \\
	0 & 0 & \ddots & 1 & 0 \\
	0 & 0 & \dots & 1 & 0 \\
	0 & 0 & \dots & 0 & 1 \\
	\end{pmatrix} \cdot \begin{pmatrix} 
	1 & 0 & \dots & 0 & 0 \\
	0 & 1 & \dots & 0 & 0 \\
	0 & 0 & \ddots & 1 & 0 \\
	0 & 0 & \dots & 1 & 0 \\
	0 & 0 & \dots & 0 & 1 \\
	\end{pmatrix}^{m}
	\quad
	$$
(by the assumption)
$$
	=\begin{pmatrix} 
	1 & 0 & \dots & 0 & 0 \\
	0 & 1 & \dots & 0 & 0 \\
	0 & 0 & \ddots & 1 & 0 \\
	0 & 0 & \dots & 1 & 0 \\
	0 & 0 & \dots & 0 & 1 \\
	\end{pmatrix} \cdot \begin{pmatrix} 
	1 & 0 & \dots & 0 & 0 \\
	0 & 1 & \dots & 0 & 0 \\
	0 & 0 & \ddots & m & 0 \\
	0 & 0 & \dots & 1 & 0 \\
	0 & 0 & \dots & 0 & 1 \\
	\end{pmatrix}=\begin{pmatrix} 
	1 & 0 & \dots & 0 & 0 \\
	0 & 1 & \dots & 0 & 0 \\
	0 & 0 & \ddots & m+1 & 0 \\
	0 & 0 & \dots & 1 & 0 \\
	0 & 0 & \dots & 0 & 1 \\
	\end{pmatrix}
	\quad
	$$

\newpage
\underline{\textbf{Commutators of elementary matrices}} \newline

Let \( \{ E_{i,j} \}_{i<j} \) be the set of all elementary matrices of this form. \newline

\textbf{Exercise}
\( (a_{l,m})=E_{i,j}^{-1} \) is the matrix with \( 1 \) on the main diagonal, and \( -1 \) in \( a_{i,j} \) \newline
\textbf{Proof}
We can see that directly from the fact that in order to have \newline \( (c_{l,m})=(a_{l,m}) \cdot (b_{l,m})=E_{i,j} \cdot E_{i,j}^{-1}=I \), \newline
we need to have \( c_{i,j}=0 \), which means that adding row \( j \) to row \( i \), in \( E_{i,j}^{-1} \) (by the left multiplication of \( E_{i,j} \)) \newline
must give \( a_{i,j}+b_{i,j}=c_{i,j}=0 \Rightarrow b_{i,j}=-a_{i,j}=-1 \) \newline

\textbf{Exercise}
\( [E_{i,j},E_{j,k}]=E_{i,k} \) \newline
\textbf{Proof}
\( E_{i,j} \) is operating from left on \( E_{j,k} \) by addition of row \( j \) to row \( i \),
so, the product matrix, \( (a_{l,m}=E_{i,j} \cdot E_{j,k} \) has \( 1 \) on the main diagonal and in \( a_{j,k},a_{i,j},a_{i,k} \) \newline
\( E_{i,j}^{-1} \) is operating from left on \( E_{j,k}^{-1} \) by subtraction of row \( j \) from row \( i \),
so, the product matrix, \( (b_{l,m}=E_{i,j}^{-1} \cdot E_{j,k}^{-1} \) has \( 1 \) on the main diagonal and in \( b_{i,k} \), \newline
and \( -1 \) in \( b_{j,k},b_{i,j} \) \newline
Multiplying \( (a_{l,m} \cdot (b_{l,m}) \) yields a product matrix, \( (c_{l,m} \) with \( 1 \) on the main diagonal, and, \newline
since \( a_{i,i}=a_{i,j}=a_{i,k}=1 \), with all other cells in row \( j \) being \(0 \),
and since  \( b_{i,k}=b_{k,k}=1 \), and \( b_{j,k}=-1 \), multiplying row \( (a_{l,m})_{i} \) by column \( (b_{l,m})_{k} \) yields
the value \( c_{i,k}=b_{i,k}+b_{j,k}+b_{k,k}=1-1+1=1 \) \newline
We can see that multiplying \( (a_{l,m})_{i} \cdot (b_{l,m})_{j} \)
yields \( c_{i,j}=a_{i,i} \cdot b_{i,j}+a_{i,j} \cdot b_{j,j}=1 \cdot -1+1 \cdot 1=1-1=0 \) \newline
And, we can see that multiplying \( (a_{l,m})_{j} \cdot (b_{l,m})_{k} \)
yields \( c_{j,k}=a_{j,j} \cdot b_{j,k}+a_{j,k} \cdot b_{k,k}=1 \cdot -1+1 \cdot 1=1-1=0 \)

For example, \( n=4 \), \newline
$$ E_{1,2} \cdot E_{2,3}=\begin{pmatrix} 
	1 & 1 & 0 & 0 \\
	0 & 1 & 0 & 0 \\
	0 & 0 & 1 & 0 \\
	0 & 0 & 0 & 1 \\
	\end{pmatrix} \cdot \begin{pmatrix} 
	1 & 0 & 0 & 0 \\
	0 & 1 & 1 & 0 \\
	0 & 0 & 1 & 0 \\
	0 & 0 & 0 & 1 \\
	\end{pmatrix}=\begin{pmatrix} 
	1 & 1 & 1 & 0 \\
	0 & 1 & 1 & 0 \\
	0 & 0 & 1 & 0 \\
	0 & 0 & 0 & 1 \\
	\end{pmatrix} $$
$$  E_{1,2}^{-1} \cdot E_{2,3}^{-1}=\begin{pmatrix} 
	1 & -1 & 0 & 0 \\
	0 & 1 & 0 & 0 \\
	0 & 0 & 1 & 0 \\
	0 & 0 & 0 & 1 \\
	\end{pmatrix} \cdot \begin{pmatrix} 
	1 & 0 & 0 & 0 \\
	0 & 1 & -1 & 0 \\
	0 & 0 & 1 & 0 \\
	0 & 0 & 0 & 1 \\
	\end{pmatrix}=\begin{pmatrix} 
	1 & -1 & 1 & 0 \\
	0 & 1 & -1 & 0 \\
	0 & 0 & 1 & 0 \\
	0 & 0 & 0 & 1 \\
	\end{pmatrix} \newline
$$
$$  [E_{1,2} \cdot E_{2,3}]=E_{1,2} \cdot E_{2,3} \cdot E_{1,2}^{-1} \cdot E_{2,3}^{-1}=$$
$$=\begin{pmatrix} 
	1 & 1 & 1 & 0 \\
	0 & 1 & 1 & 0 \\
	0 & 0 & 1 & 0 \\
	0 & 0 & 0 & 1 \\
	\end{pmatrix} \cdot \begin{pmatrix} 
	1 & -1 & 1 & 0 \\
	0 & 1 & -1 & 0 \\
	0 & 0 & 1 & 0 \\
	0 & 0 & 0 & 1 \\
	\end{pmatrix}=\begin{pmatrix} 
	1 & 0 & 1 & 0 \\
	0 & 1 & 0 & 0 \\
	0 & 0 & 1 & 0 \\
	0 & 0 & 0 & 1 \\
\end{pmatrix}=E_{1,3}$$

\textbf{Exercise}
\( [E_{i,j},E_{l,k}]=I \), where \( j \neq l \) \newline
\textbf{Proof}
\( E_{i,j} \) is operating from left on \( E_{l,k} \) by addition of row \( j \) to row \( i \),
so, the product matrix, \( (a_{n,m}=E_{i,j} \cdot E_{l,k} \) has \( 1 \) on the main diagonal and in \( a_{l,k},a_{i,j} \) \newline
\( E_{i,j}^{-1} \) is operating from left on \( E_{l,k}^{-1} \) by subtraction of row \( j \) from row \( i \),
so, the product matrix, \( (b_{n,m}=E_{i,j}^{-1} \cdot E_{l,k}^{-1} \) has \( 1 \) on the main diagonal, \newline
and \( -1 \) in \( b_{l,k},b_{i,j} \) \newline
We can see that multiplying \( (a_{n,m})_{i} \cdot (b_{n,m})_{j} \)
yields \( c_{i,j}=a_{i,i} \cdot b_{i,j}+a_{i,j} \cdot b_{j,j}=1 \cdot -1+1 \cdot 1=1-1=0 \) \newline
And, we can see that multiplying \( (a_{n,m})_{l} \cdot (b_{n,m})_{k} \)
yields \( c_{l,k}=a_{l,l} \cdot b_{l,k}+a_{l,k} \cdot b_{k,k}=1 \cdot -1+1 \cdot 1=1-1=0 \)

\end{document}
