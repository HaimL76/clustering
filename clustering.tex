\documentclass[12pt]{article}
\makeatletter
\newcommand*{\rom}[1]{\expandafter\@slowromancap\romannumeral #1@}
\makeatother
\usepackage{amsfonts}
\usepackage{amsmath}
\usepackage{tikz-cd}
\usepackage{xcolor}
\usepackage{algorithm}
\usepackage{algorithmic}

\begin{document}
\textbf{Clustering Optimization using k-means}

\textbf{Preface}
In this paper, we shall demonstrate two approaches for the optimizing the k-means algorithm for clustering. \newline

There are two kinds of clustering that we would like to consider, \newline
\underline{which may affect the algorithm that we run}, \newline \newline
\textbf{1. Clustering of scattered details} \newline
In this kind of clustering, we try to identify clusters from details that are scattered along the image.
This can have various applications, such as \newline \textbf{Military Intelligence} (clusters of troops, aircraft, armored vehicles), \newline
\textbf{Nature Science} (clustered structures of birds, insects, fish) and many more. \newline \newline
\textbf{2. Identification of natural clusters} \newline
In this kind of clustering, we try to identify large objects, from samples found in the imagery. A major example for this
kind of clustering would be face recognition, where we have many features in some photo, and we try to isolate and
identify the face of a person, or of several people. \newline

\textbf{The basic k-means algorithm}
One of the classic algorithms for clustering is the k-means algorithm. This algorithm is based on a very simple concept
of acquiring initial data, then adjusting this data until the algorithm stables. \newline
This algorithm is called k-means, because we are trying to find \( k \in \mathbb{N} \) clusters, which are supposed to
give the optimal clustering, because each cluster has a center point, which is the mean of all the points that are grouped
together in this cluster. We call this cluster, or the center point of this cluster, a \textbf{centroid}. \newline

The basic description of this algorithm, for a given k, is,

\begin{algorithm}
\caption{Calculate k-means}
\begin{algorithmic} 
\REQUIRE 
\ENSURE 
\STATE $L \leftarrow size(samples)$
\STATE $i \leftarrow 1$
\WHILE{$i \neq L$}
\STATE $s \leftarrow samples[i]$

\STATE $j \leftarrow 1$
\WHILE{$j \neq k$}
\STATE $c \leftarrow centroids[j]$
\STATE $dx \leftarrow s.x - c.x$
\STATE $dy \leftarrow s.y - c.y$

\STATE $d2 \leftarrow dx^2+dy^2$
\ENDWHILE
\ENDWHILE
\IF{$n < 0$}
\STATE $X \leftarrow 1 / x$
\STATE $N \leftarrow -n$
\ELSE
\STATE $X \leftarrow x$
\STATE $N \leftarrow n$
\ENDIF
\WHILE{$N \neq 0$}
\IF{$N$ is even}
\STATE $X \leftarrow X \times X$
\STATE $N \leftarrow N / 2$
\ELSE[$N$ is odd]
\STATE $y \leftarrow y \times X$
\STATE $N \leftarrow N - 1$
\ENDIF
\ENDWHILE
\end{algorithmic}
\end{algorithm}










\end{document}
